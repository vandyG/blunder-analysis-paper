% LaTeX version of project_proposal.md
\documentclass[11pt]{article}
\usepackage[utf8]{inputenc}
\usepackage[T1]{fontenc}
\usepackage{lmodern}
\usepackage{microtype}
\usepackage{geometry}
\usepackage{hyperref}
\usepackage{amsmath,amssymb}
\usepackage{enumitem}
\geometry{margin=1in}

\title{Chess Blunder Analysis Using Multivariate and Linear Algebraic Approaches}
\author{Vandit Goel \\
Student ID: 1002245699 \\
\texttt{vxg5699@mavs.uta.edu}}
\date{September 28, 2025}

\begin{document}
\maketitle

\begin{abstract}
This project investigates the phenomenon of blunders in chess by analyzing when and why players make them, with a focus on combining multiple explanatory factors by analyzing large-scale game data from the publicly available Lichess database. Each move will be evaluated using the Stockfish engine to quantify shifts in evaluation and win/draw/loss probabilities, which will serve as a baseline for identifying blunders. Alongside these engine-derived features, additional factors such as the sharpness of the position, the type of piece moved, and temporal factors such as time remaining and time spent per move will be incorporated into the analysis. By combining these dimensions, the study aims to capture the complex interplay between position difficulty, human decision-making under pressure, and move selection errors.
\end{abstract}

\section{Introduction}

The project draws on large-scale chess data and modern engine evaluations to study human mistakes (``blunders'') in chess. The Lichess database, which publishes on the order of tens of millions of games per month, will be the primary data source. For each move in a sampled set of games we will run Stockfish to obtain a high-quality evaluation and estimated win/draw/loss probabilities. Substantial negative shifts in engine evaluation following a move will be used to flag blunders; these engine signals will form the core, objective baseline for the remaining analysis.

\section{Objectives}
\begin{itemize}[nosep]
  \item Quantify and classify blunders using Stockfish evaluation shifts and outcome probabilities.
  \item Augment engine-derived signals with contextual features: position sharpness, piece type moved, time remaining, and time spent per move.
  \item Encode positions and features as matrices or tensors to enable linear-algebraic and multivariate analysis.
  \item Build predictive models (including neural networks) to estimate blunder probability for a given position and context.
  \item Apply dimensionality-reduction, matrix factorizations, and clustering on kernel matrices to group and interpret blunder types.
\end{itemize}

\section{Methodology}

\subsection{Data collection}
We will sample games from the Lichess monthly dumps. Each game's moves will be re-evaluated by Stockfish at a fixed depth or time control to obtain consistent engine evaluations and win/draw/loss probabilities.

\subsection{Feature engineering}
Features will include: engine evaluation shifts (quantitative blunder signals), measures of position sharpness (e.g., variance of candidate move evaluations or tactical motifs), categorical indicators such as piece type moved, and temporal features like remaining clock time and per-move time usage. Positions will be encoded into matrix/tensor forms (for example board-planes or feature tensors) that are compatible with linear algebra operations and neural network inputs.

\subsection{Modeling and analysis}
We will use matrix factorizations (SVD, PCA) for dimensionality reduction and interpretability, kernel-based clustering to group similar blunder phenomena, and supervised learning models (including neural networks) to predict blunder probabilities. Optimization methods and regularization will be used to ensure generalization.

\section{Expected contributions}
This experimental study aims to provide:
\begin{itemize}[nosep]
  \item A reproducible pipeline for labeling blunders using engine evaluations and contextual features.
  \item A comparative analysis of classical linear-algebraic techniques (factorizations, kernel clustering) and modern supervised models for predicting human errors.
  \item Insights into which contextual factors most strongly influence blunder likelihood (time pressure, position sharpness, piece moved, etc.).
\end{itemize}

\section{Risks and limitations}
The project is computationally intensive because re-evaluating many moves with Stockfish at sufficient depth requires substantial compute. Additionally, feature engineering choices (how to measure sharpness, how to encode positions) may strongly influence results. We will mitigate these risks by: sampling thoughtfully, experimenting with multiple evaluation depths, and clearly documenting feature definitions and preprocessing choices.

\section{Timeline (high level)}
\begin{enumerate}[nosep]
  \item Data acquisition and preprocessing: sample games, run engine evaluations, compute base features.
  \item Exploratory analysis and feature selection: initial dimensionality reduction and clustering to identify interesting blunder classes.
  \item Modeling: train and validate predictive models, iterate on features and regularization.
  \item Evaluation and write-up: interpret models, assess predictive performance, and document reproducible code and experiments.
\end{enumerate}

\section{Contact}
For questions or collaboration, contact Vandit Goel at \texttt{vxg5699@mavs.uta.edu}.

\vfill
\noindent{\footnotesize Generated from the project proposal markdown on September 28, 2025.}

\end{document}
