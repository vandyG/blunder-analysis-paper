% Results section for Chess Blunder Analysis
\section{Results}\label{sec:results}

This section presents empirical findings from the processed Lichess sample. Unless stated otherwise, an ``error'' denotes any move classified as Inaccuracy, Mistake, or Blunder via the winning chance delta thresholds (Section~\ref{sec:problem_formulation}).

\subsection{Error Rates Across Elo and Time Controls}
Error prevalence generally declines with increasing Elo across time controls (Figures~\ref{fig:elo-blitz}--\ref{fig:elo-standard}), but the pattern of decline depends strongly on rating strata and time control. The steepest slope of decline occurs among the very highest-rated players (Elo $>2600$), who show marked reductions in error rates relative to slightly lower tiers. Longer time formats (Rapid and Standard) exhibit a larger decline in errors across Elo than Bullet; Bullet still shows a noticeable improvement for the top ($>2600$) players but the gradient is less pronounced than in longer formats.

Key observations:
\begin{itemize}
  \item \textbf{Bullet (Figure~\ref{fig:elo-bullet})}: Error proportions are elevated at lower Elo and improve with rating, suggesting pattern recognition and practical skills partially compensate for extreme time pressure.
  \item \textbf{Blitz (Figure~\ref{fig:elo-blitz})}: Shows a sharper decline across Elo than Bullet for higher rating bands, reflecting that modestly longer decision time reduces time-induced noise.
  \item \textbf{Rapid (Figure~\ref{fig:elo-rapid})}: Transitional regime; error rates fall more substantially with Elo than in Bullet/Blitz with higher-rated players.
  \item \textbf{Standard (Figure~\ref{fig:elo-standard})}: Distinct profile: higher-rated players (Elo >2400) make significantly fewer errors, while lower-rated players continue to exhibit relatively high error rates, producing a more bimodal-like relationship between Elo and error prevalence.
\end{itemize}


\subsection{Error Rates as a Function of Sharpness}
Figure~\ref{fig:error-sharpness} summarizes how the empirical proportion of moves labeled as Inaccuracy, Mistake, or Blunder varies with the sharpness score $S_t = P_t^{\mathrm{win}} + P_t^{\mathrm{loss}}$ defined in Section~\ref{sec:problem_formulation}. Recall that $S_t$ is close to $0$ in positions with essentially no winning chances for either side and approaches $1$ when the engine believes a decisive result is likely for one player.

Across most of the range, higher sharpness is associated with a higher fraction of moves being flagged as errors. All three curves (Inaccuracy, Mistake, Blunder) exhibit an overall upward trend from low $S_t$ toward the mid-to-high range, consistent with the interpretation of sharp positions as ``easy to mess up'' for either side. Positions where both sides still have realistic winning chances---reflected by substantial mass in $P_t^{\mathrm{win}}$ and/or $P_t^{\mathrm{loss}}$---lead to more frequent deviations from engine-best play.

The effect is strongest for inaccuracies: their proportion rises steeply as $S_t$ increases from very low values and peaks around intermediate-to-high sharpness, suggesting that players often make small but non-fatal errors in tactically volatile positions. Mistakes and blunders show a similar but less steep monotone increase, indicating that severe errors also become more common as positions become sharper, though the absolute rates remain lower than for inaccuracies.

At the extreme ends of the sharpness scale the curves drop, but with an important nuance at the low end. All three series exhibit a dip in error proportion around $S_t \approx 0.1$, where positions are low-sharpness and players rarely commit large mistakes. In contrast, the blunder curve spikes very close to $S_t = 0$, suggesting that some moves made in objectively "dead" or trivially equal positions receive almost no attention and are played carelessly, leading to disproportionately many catastrophic errors. The drop near $S_t \approx 1$ is less structural and should be interpreted cautiously. Very sharp, almost-decided positions are rare in the sample, so the corresponding empirical proportions are based on few observations and are sensitive to binning choices; in addition, once a position is nearly winning or losing, many moves may be equally decisive in the engine's view, which can dampen measured error rates despite high nominal sharpness.


\subsection{Move-level time usage (time\_ratio)}
The per-move time usage is summarized by the ratio
\[
	\text{time\_ratio}_t = \frac{(\text{clock}_{t-1} + \text{increment}) - \text{clock}_t}{\text{clock}_{t-1} + \text{increment}},
\]
which measures the fraction of the available clock at move $t$ that was consumed making that move. Values near $0$ indicate very quick moves (including pre-moves or routine, low-effort moves), while values near $1$ indicate that almost the entire available time allotment was used.

Empirically (Figure~\ref{fig:time_ratio}), the proportion of moves classified as errors increases with `time\_ratio` across all fast time controls. Key patterns:

\begin{itemize}
  \item \textbf{Monotonic increase:} Error proportion rises from low `time\_ratio` (quick moves) up through intermediate values, with a clear upward trend for Bullet and Blitz and a milder slope for Rapid. This suggests that moves made after spending a larger share of available time are, on average, more likely to be errors than very quick, routine moves.
  \item \textbf{Stronger effect in faster controls:} The steepest increase appears in Bullet and Blitz, consistent with time-pressure amplifying decision noise. Rapid shows a similar shape but lower overall error proportions.
  \item \textbf{Extreme-value volatility:} At very high `time\_ratio` (close to 1.0) the series becomes noisy, with occasional spikes and abrupt drops. Those features are at least partly sampling noise: very few moves consume essentially the entire allotment in some bins, producing high variance in the empirical proportion.
  \item \textbf{Interpretation hypotheses:} Low `time\_ratio` often corresponds to low-complexity positions or automatic pattern play (hence low errors). Larger `time\_ratio` can represent either (i) genuinely harder positions requiring long calculation (increasing error risk despite more time), or (ii) extreme time pressure where players are forced to use most of their remaining clock repeatedly and still make mistakes.
\end{itemize}

For modeling, `time\_ratio` should be included as a non-linear predictor (e.g., spline or low-order polynomial) and interacted with time-control category and Elo. Additionally, control for move phase and position-sharpness metrics to disambiguate whether high `time\_ratio` reflects complexity or time-scramble. Finally, use binning or variance-aware smoothing and report counts per bin to avoid over-interpreting noisy extremes.
\subsection{Piece-Type Error Distribution}
Figure~\ref{fig:piece-error} reports error composition by moved piece. Knights and Queens show relatively higher severe error proportions compared to Pawns. Hypotheses:
\begin{itemize}
  \item \textbf{Knights}: High branching tactical motifs; miscalculation of knight forks and intermediate squares leads to larger evaluation drops.
  \item \textbf{Queens}: Centralization and overextension errors produce large tactical liabilities when tempo is misjudged.
  \item \textbf{Pawns}: Majority of pawn moves produce incremental positional shifts, hence lower severe error proportion; inaccuracies dominate rather than blunders.
  \item \textbf{Rooks/Bishops}: Intermediate; their long-range nature creates both strategic and tactical opportunities but fewer catastrophic single-move losses than queen blunders.
\end{itemize}

\subsection{Principal Component Analysis of Feature Space}
\label{sec:pca_results}

To understand the intrinsic dimensionality of the feature space and identify dominant patterns of variation, we applied Incremental PCA (Section~\ref{sec:ipca}) to the standardized feature matrix excluding the target variable \texttt{is\_error}. The analysis used 10 principal components and processed $12,540,813$ move records in batches of 10,000 observations.

\subsubsection{Explained Variance}
Table~\ref{tab:pca_variance} reports the explained variance ratio for each of the 10 principal components. The first component captures 10.73\% of total variance, the second 8.29\%, and subsequent components capture progressively smaller fractions. The cumulative explained variance reaches approximately 62.3\% after 10 components.

\begin{table}[ht]
\centering
\begin{tabular}{c|c|c}
\hline
\textbf{Component} & \textbf{Variance Ratio} & \textbf{Cumulative Variance} \\
\hline
PC1  & 0.1073 & 0.1073 \\
PC2  & 0.0829 & 0.1902 \\
PC3  & 0.0755 & 0.2657 \\
PC4  & 0.0572 & 0.3229 \\
PC5  & 0.0566 & 0.3795 \\
PC6  & 0.0550 & 0.4345 \\
PC7  & 0.0519 & 0.4864 \\
PC8  & 0.0502 & 0.5365 \\
PC9  & 0.0435 & 0.5801 \\
PC10 & 0.0430 & 0.6231 \\
\hline
\end{tabular}
\caption{Explained variance ratios for the first 10 principal components.}
\label{tab:pca_variance}
\end{table}

\subsubsection{Interpretation of Low Explained Variance}
The relatively low individual and cumulative explained variance ratios indicate that the feature space does not exhibit strong low-rank structure. Several factors contribute to this finding:

\begin{itemize}
    \item \textbf{Feature heterogeneity:} The feature set combines diverse modalities---temporal metrics (\texttt{game\_time}, \texttt{time\_ratio}), evaluation scores (\texttt{cp\_score}, WDL probabilities), categorical encodings (time control, piece type), and player skill (\texttt{player\_elo}). These features capture fundamentally different aspects of the game state and player behavior, and there is no reason to expect them to align along a small number of shared axes.
    
    \item \textbf{High intrinsic dimensionality:} Chess positions and move contexts are inherently high-dimensional phenomena. The strategic, tactical, temporal, and psychological factors that determine blunder likelihood operate on distinct, weakly correlated dimensions. Standard PCA seeks linear combinations that maximize variance, but the absence of dominant global directions suggests that blunder patterns are better described by localized, non-linear manifold structure or feature interactions rather than simple linear projections.
    
    \item \textbf{Exclusion of positional features:} The 768-dimensional bitboard encoding was excluded from this analysis due to memory constraints (Section~\ref{sec:fen_encoding}). Board structure likely contains rich, compressible spatial patterns (e.g., pawn chains, king safety configurations) that could exhibit stronger low-rank structure. The current analysis is limited to evaluation-derived and metadata features, which are more abstract and less redundant than raw piece placements.
    
    \item \textbf{Noise and stochasticity:} Human move choices contain irreducible noise from cognitive limitations, time pressure, and psychological factors. This noise inflates the effective dimensionality and prevents variance from concentrating in a few components.
\end{itemize}

Despite the modest explained variance, the PCA projection remains useful for several purposes:
\begin{enumerate}
    \item \textbf{Decorrelation:} The principal components are orthogonal by construction, eliminating multicollinearity for downstream linear models.
    \item \textbf{Baseline for comparison:} The low explained variance establishes that simple linear dimensionality reduction is insufficient for this task, motivating more sophisticated approaches such as autoencoders, kernel PCA, or direct supervised feature learning via neural networks.
\end{enumerate}

\subsubsection{Scatter Plot of First Two Components}
Figure~\ref{fig:pca_scatter} shows a scatter plot of the first two principal components, with points colored by the binary \texttt{is\_error} label. The two classes (error vs non-error) overlap substantially in the PC1--PC2 plane, confirming that linear projections do not achieve clean separation. This overlap is consistent with the low explained variance and suggests that blunder prediction will require non-linear classifiers or richer feature representations (e.g., convolutional embeddings of board positions) to capture the complex decision boundaries separating errors from correct moves.

% Placeholder for PCA scatter plot figure
% \begin{figure}[ht]
%   \centering
%   \includegraphics[width=0.65\textwidth]{../data/images/pca_scatter.png}
%   \caption{Scatter plot of first two IPCA components, colored by error class. Substantial overlap indicates that linear projections do not cleanly separate errors from non-errors.}
%   \label{fig:pca_scatter}
% \end{figure}

% ---------------- Figures ----------------
\begin{figure}[ht]
  \centering
  \begin{subfigure}[b]{0.35\textwidth}
    \centering
    \includegraphics[width=\linewidth]{../data/images/error_vs_elo_bullet.png}
    \caption{Bullet}\label{fig:elo-bullet}
  \end{subfigure}
  \hfill
  \begin{subfigure}[b]{0.35\textwidth}
    \centering
    \includegraphics[width=\linewidth]{../data/images/error_vs_elo_blitz.png}
    \caption{Blitz}\label{fig:elo-blitz}
  \end{subfigure}

  \vspace{6pt}

  \begin{subfigure}[b]{0.35\textwidth}
    \centering
    \includegraphics[width=\linewidth]{../data/images/error_vs_elo_rapid.png}
    \caption{Rapid}\label{fig:elo-rapid}
  \end{subfigure}
  \hfill
  \begin{subfigure}[b]{0.35\textwidth}
    \centering
    \includegraphics[width=\linewidth]{../data/images/errors_vs_elo_standard.png}
    \caption{Standard}\label{fig:elo-standard}
  \end{subfigure}

  \caption{Elo-binned error proportions by time control: Bullet, Blitz, Rapid, and Standard.}
  \label{fig:elo-grid}
\end{figure}

\begin{figure}[h]
    \centering
    \begin{subfigure}[b]{0.3\textwidth}
        \centering
        \includegraphics[width=\linewidth]{../data/images/error_vs_time_ratio_bullet.png}
        \caption{Bullet}
    \end{subfigure}
    \hfill
    \begin{subfigure}[b]{0.3\textwidth}
        \centering
        \includegraphics[width=\linewidth]{../data/images/error_vs_time_ratio_blitz.png}
        \caption{Blitz}
    \end{subfigure}
    \hfill
    \begin{subfigure}[b]{0.3\textwidth}
        \centering
        \includegraphics[width=\linewidth]{../data/images/error_vs_time_ratio_rapid.png}
        \caption{Rapid}
    \end{subfigure}
    \caption{Proportion of moves classified as errors by `time\_ratio` for Bullet, Blitz, and Rapid time controls.}
    \label{fig:time_ratio}
\end{figure}

\begin{figure}[h]
  \centering
  \includegraphics[width=0.80\textwidth]{../data/images/error_vs_sharpness.png}
  \caption{Proportion of moves classified as Inaccuracy, Mistake, or Blunder by sharpness score $S_t = P_t^{\mathrm{win}} + P_t^{\mathrm{loss}}$.}\label{fig:error-sharpness}
\end{figure}

\begin{figure}[ht]
  \centering
  \includegraphics[width=0.68\textwidth]{../data/images/error_vs_piece.png}
  \caption{Error composition by moved piece type.}\label{fig:piece-error}
\end{figure}

\begin{figure}[ht]
  \centering
  \includegraphics[width=0.68\textwidth]{../data/images/pca.png}
  \caption{Principal Component Analysis.}\label{fig:pca_scatter}
\end{figure}

\clearpage

